\documentclass[12pt]{article}\usepackage[]{graphicx}\usepackage[]{color}
%% maxwidth is the original width if it is less than linewidth
%% otherwise use linewidth (to make sure the graphics do not exceed the margin)
\makeatletter
\def\maxwidth{ %
  \ifdim\Gin@nat@width>\linewidth
    \linewidth
  \else
    \Gin@nat@width
  \fi
}
\makeatother

\definecolor{fgcolor}{rgb}{0.345, 0.345, 0.345}
\newcommand{\hlnum}[1]{\textcolor[rgb]{0.686,0.059,0.569}{#1}}%
\newcommand{\hlstr}[1]{\textcolor[rgb]{0.192,0.494,0.8}{#1}}%
\newcommand{\hlcom}[1]{\textcolor[rgb]{0.678,0.584,0.686}{\textit{#1}}}%
\newcommand{\hlopt}[1]{\textcolor[rgb]{0,0,0}{#1}}%
\newcommand{\hlstd}[1]{\textcolor[rgb]{0.345,0.345,0.345}{#1}}%
\newcommand{\hlkwa}[1]{\textcolor[rgb]{0.161,0.373,0.58}{\textbf{#1}}}%
\newcommand{\hlkwb}[1]{\textcolor[rgb]{0.69,0.353,0.396}{#1}}%
\newcommand{\hlkwc}[1]{\textcolor[rgb]{0.333,0.667,0.333}{#1}}%
\newcommand{\hlkwd}[1]{\textcolor[rgb]{0.737,0.353,0.396}{\textbf{#1}}}%

\usepackage{framed}
\makeatletter
\newenvironment{kframe}{%
 \def\at@end@of@kframe{}%
 \ifinner\ifhmode%
  \def\at@end@of@kframe{\end{minipage}}%
  \begin{minipage}{\columnwidth}%
 \fi\fi%
 \def\FrameCommand##1{\hskip\@totalleftmargin \hskip-\fboxsep
 \colorbox{shadecolor}{##1}\hskip-\fboxsep
     % There is no \\@totalrightmargin, so:
     \hskip-\linewidth \hskip-\@totalleftmargin \hskip\columnwidth}%
 \MakeFramed {\advance\hsize-\width
   \@totalleftmargin\z@ \linewidth\hsize
   \@setminipage}}%
 {\par\unskip\endMakeFramed%
 \at@end@of@kframe}
\makeatother

\definecolor{shadecolor}{rgb}{.97, .97, .97}
\definecolor{messagecolor}{rgb}{0, 0, 0}
\definecolor{warningcolor}{rgb}{1, 0, 1}
\definecolor{errorcolor}{rgb}{1, 0, 0}
\newenvironment{knitrout}{}{} % an empty environment to be redefined in TeX

\usepackage{alltt}

\usepackage{amssymb,amsmath}
\usepackage{enumerate}
\usepackage{float}
\usepackage{verbatim}
\usepackage{setspace}
\usepackage{multicol}

%% LaTeX margin settings:
  \setlength{\textwidth}{7.0in}
\setlength{\textheight}{9in}
\setlength{\oddsidemargin}{-.5in}
\setlength{\evensidemargin}{0in}
\setlength{\topmargin}{-1.5cm}

%% tell knitr to use smaller font for code chunks
\def\fs{\footnotesize}
\def\R{{\sf R}}
\newcommand{\bfbeta}{\mbox{\boldmath $\beta$}}
\newcommand{\bfD}{\mbox{\boldmath $D$}}
\newcommand{\bfL}{\mbox{\boldmath $L$}}
\newcommand{\bfR}{\mbox{\boldmath $R$}}
\newcommand{\bfmu}{\mbox{\boldmath $\mu$}}
\newcommand{\bfv}{\mbox{\boldmath $V$}}
\newcommand{\bfX}{\mbox{\boldmath $X$}}
\newcommand{\bfy}{\mbox{\boldmath $y$}}
\newcommand{\bfb}{\mbox{\boldmath $b$}}
\newcommand{\ytil}{\mbox{$\tilde{y}$}}
\IfFileExists{upquote.sty}{\usepackage{upquote}}{}
\begin{document}


  
  
  \begin{center}
\large{Sampling: Final Exam} \\
Leslie Gains-Germain
\end{center}

\begin{doublespacing}






\begin{enumerate}

\item

\item

\item \begin{enumerate}

\item In the first table, all of the possible samples are shown, along with the probability of drawing each sample (found with replacement). The inclusion probabilities for units I, II, III, IV, V, and VI are shown in the table on the right. The R code for computing these values are shown below.

\begin{multicols}{2}
\begin{table}[H]
\begin{tabular}{ccc}
Sample & Units & $P(S = s)$ \\
\hline
1 & 1,2 & 0.04889 \\
2 & 1,3 & 0.07334 \\
3 & 1,4 & 0.01497 \\
4 & 1,5 & 0.01834 \\
5 & 1,6 & 0.03056 \\
6 & 2,3 & 0.05867 \\
7 & 2,4 & 0.01198 \\
8 & 2,5 & 0.01467 \\
9 & 2,6 & 0.02445 \\
10 & 3,4 & 0.01797 \\
11 & 3,5 & 0.02200 \\
12 & 3,6 & 0.03667 \\
13 & 4,5 & 0.004492 \\
14 & 4,6 & 0.007487 \\
15 & 5,6 & 0.009168 \\
\hline
\end{tabular}
\end{table}

\begin{table}[H]
\begin{tabular}{cc}
Unit & $p_i$ \\
\hline
I & 0.1861 \\
II & 0.1098 \\
III & 0.2087 \\
IV & 0.0569 \\
V & 0.06867 \\
VI & 0.1083 \\
\hline
\end{tabular}
\end{table}
\end{multicols}

\begin{singlespace}
\begin{knitrout}\footnotesize
\definecolor{shadecolor}{rgb}{0.969, 0.969, 0.969}\color{fgcolor}\begin{kframe}
\begin{alltt}
\hlstd{a1} \hlkwb{<-} \hlnum{200}
\hlstd{a2} \hlkwb{<-} \hlnum{160}
\hlstd{a3} \hlkwb{<-} \hlnum{240}
\hlstd{a4} \hlkwb{<-} \hlnum{49}
\hlstd{a5} \hlkwb{<-} \hlnum{60}
\hlstd{a6} \hlkwb{<-} \hlnum{100}
\hlstd{total.area} \hlkwb{<-} \hlstd{a1} \hlopt{+} \hlstd{a2} \hlopt{+} \hlstd{a3} \hlopt{+} \hlstd{a4} \hlopt{+} \hlstd{a5} \hlopt{+} \hlstd{a6}
\hlstd{p12} \hlkwb{<-} \hlstd{a1}\hlopt{*}\hlstd{a2}\hlopt{/}\hlstd{total.area}\hlopt{^}\hlnum{2}
\hlstd{p13} \hlkwb{<-} \hlstd{a1}\hlopt{*}\hlstd{a3}\hlopt{/}\hlstd{total.area}\hlopt{^}\hlnum{2}
\hlstd{p14} \hlkwb{<-} \hlstd{a1}\hlopt{*}\hlstd{a4}\hlopt{/}\hlstd{total.area}\hlopt{^}\hlnum{2}
\hlstd{p15} \hlkwb{<-} \hlstd{a1}\hlopt{*}\hlstd{a5}\hlopt{/}\hlstd{total.area}\hlopt{^}\hlnum{2}
\hlstd{p16} \hlkwb{<-} \hlstd{a1}\hlopt{*}\hlstd{a6}\hlopt{/}\hlstd{total.area}\hlopt{^}\hlnum{2}
\hlstd{p23} \hlkwb{<-} \hlstd{a2}\hlopt{*}\hlstd{a3}\hlopt{/}\hlstd{total.area}\hlopt{^}\hlnum{2}
\hlstd{p24} \hlkwb{<-} \hlstd{a2}\hlopt{*}\hlstd{a4}\hlopt{/}\hlstd{total.area}\hlopt{^}\hlnum{2}
\hlstd{p25} \hlkwb{<-} \hlstd{a2}\hlopt{*}\hlstd{a5}\hlopt{/}\hlstd{total.area}\hlopt{^}\hlnum{2}
\hlstd{p26} \hlkwb{<-} \hlstd{a2}\hlopt{*}\hlstd{a6}\hlopt{/}\hlstd{total.area}\hlopt{^}\hlnum{2}
\hlstd{p34} \hlkwb{<-} \hlstd{a3}\hlopt{*}\hlstd{a4}\hlopt{/}\hlstd{total.area}\hlopt{^}\hlnum{2}
\hlstd{p35} \hlkwb{<-} \hlstd{a3}\hlopt{*}\hlstd{a5}\hlopt{/}\hlstd{total.area}\hlopt{^}\hlnum{2}
\hlstd{p36} \hlkwb{<-} \hlstd{a3}\hlopt{*}\hlstd{a6}\hlopt{/}\hlstd{total.area}\hlopt{^}\hlnum{2}
\hlstd{p45} \hlkwb{<-} \hlstd{a4}\hlopt{*}\hlstd{a5}\hlopt{/}\hlstd{total.area}\hlopt{^}\hlnum{2}
\hlstd{p46} \hlkwb{<-} \hlstd{a4}\hlopt{*}\hlstd{a6}\hlopt{/}\hlstd{total.area}\hlopt{^}\hlnum{2}
\hlstd{p56} \hlkwb{<-} \hlstd{a5}\hlopt{*}\hlstd{a6}\hlopt{/}\hlstd{total.area}\hlopt{^}\hlnum{2}
\hlstd{pi.1} \hlkwb{<-} \hlkwd{sum}\hlstd{(p12, p13, p14, p15, p16)}
\hlstd{pi.2} \hlkwb{<-} \hlkwd{sum}\hlstd{(p12, p23, p24, p25, p26)}
\hlstd{pi.3} \hlkwb{<-} \hlkwd{sum}\hlstd{(p13, p23, p34, p35, p36)}
\hlstd{pi.4} \hlkwb{<-} \hlkwd{sum}\hlstd{(p14, p24, p34, p45, p46)}
\hlstd{pi.5} \hlkwb{<-} \hlkwd{sum}\hlstd{(p15, p25, p35, p45, p56)}
\hlstd{pi.6} \hlkwb{<-} \hlkwd{sum}\hlstd{(p16, p26, p36, p46, p56)}
\hlstd{pi} \hlkwb{<-} \hlkwd{c}\hlstd{(pi.1, pi.2, pi.3, pi.4, pi.5, pi.6)}
\end{alltt}
\end{kframe}
\end{knitrout}
\end{singlespace}

\item The table below shows the $y_i/p_i$ values. The R code is shown below.
\begin{singlespace}
\begin{kframe}
\begin{alltt}
\hlstd{y1} \hlkwb{<-} \hlnum{248}
\hlstd{y2} \hlkwb{<-} \hlnum{204}
\hlstd{y3} \hlkwb{<-} \hlnum{305}
\hlstd{y4} \hlkwb{<-} \hlnum{49}
\hlstd{y5} \hlkwb{<-} \hlnum{78}
\hlstd{y6} \hlkwb{<-} \hlnum{126}
\hlstd{y} \hlkwb{<-} \hlkwd{c}\hlstd{(y1, y2, y3, y4, y5, y6)}
\hlstd{y.pi} \hlkwb{<-} \hlstd{y}\hlopt{/}\hlstd{pi}
\hlstd{unit} \hlkwb{<-} \hlkwd{c}\hlstd{(}\hlstr{"I"}\hlstd{,} \hlstr{"II"}\hlstd{,} \hlstr{"III"}\hlstd{,} \hlstr{"IV"}\hlstd{,} \hlstr{"V"}\hlstd{,} \hlstr{"VI"}\hlstd{)}
\hlkwd{xtable}\hlstd{(}\hlkwd{cbind.data.frame}\hlstd{(unit, y,} \hlstr{"p_i"} \hlstd{= pi,} \hlstr{"y/p_i"} \hlstd{= y.pi))}
\end{alltt}
\end{kframe}% latex table generated in R 3.2.2 by xtable 1.7-4 package
% Sat Nov 28 18:02:15 2015
\begin{table}[ht]
\centering
\begin{tabular}{rlrrr}
  \hline
 & unit & y & p\_i & y/p\_i \\ 
  \hline
1 & I & 248.00 & 0.19 & 1332.60 \\ 
  2 & II & 204.00 & 0.16 & 1285.77 \\ 
  3 & III & 305.00 & 0.21 & 1461.75 \\ 
  4 & IV & 49.00 & 0.06 & 861.16 \\ 
  5 & V & 78.00 & 0.07 & 1135.95 \\ 
  6 & VI & 126.00 & 0.11 & 1163.11 \\ 
   \hline
\end{tabular}
\end{table}

\end{singlespace}

\item Yes, Hansen Hurwitz estimation will provide an estimate of $\bar{y}_U$ with small variance because the selection probabilities $p_i$ are proportional to the $y_i$ values.

\end{enumerate}

\end{enumerate}

\end{doublespacing}

{\bf \large R code appendix}




\end{document}
