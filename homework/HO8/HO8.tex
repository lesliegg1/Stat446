\documentclass[12pt]{article}\usepackage[]{graphicx}\usepackage[]{color}
%% maxwidth is the original width if it is less than linewidth
%% otherwise use linewidth (to make sure the graphics do not exceed the margin)
\makeatletter
\def\maxwidth{ %
  \ifdim\Gin@nat@width>\linewidth
    \linewidth
  \else
    \Gin@nat@width
  \fi
}
\makeatother

\definecolor{fgcolor}{rgb}{0.345, 0.345, 0.345}
\newcommand{\hlnum}[1]{\textcolor[rgb]{0.686,0.059,0.569}{#1}}%
\newcommand{\hlstr}[1]{\textcolor[rgb]{0.192,0.494,0.8}{#1}}%
\newcommand{\hlcom}[1]{\textcolor[rgb]{0.678,0.584,0.686}{\textit{#1}}}%
\newcommand{\hlopt}[1]{\textcolor[rgb]{0,0,0}{#1}}%
\newcommand{\hlstd}[1]{\textcolor[rgb]{0.345,0.345,0.345}{#1}}%
\newcommand{\hlkwa}[1]{\textcolor[rgb]{0.161,0.373,0.58}{\textbf{#1}}}%
\newcommand{\hlkwb}[1]{\textcolor[rgb]{0.69,0.353,0.396}{#1}}%
\newcommand{\hlkwc}[1]{\textcolor[rgb]{0.333,0.667,0.333}{#1}}%
\newcommand{\hlkwd}[1]{\textcolor[rgb]{0.737,0.353,0.396}{\textbf{#1}}}%

\usepackage{framed}
\makeatletter
\newenvironment{kframe}{%
 \def\at@end@of@kframe{}%
 \ifinner\ifhmode%
  \def\at@end@of@kframe{\end{minipage}}%
  \begin{minipage}{\columnwidth}%
 \fi\fi%
 \def\FrameCommand##1{\hskip\@totalleftmargin \hskip-\fboxsep
 \colorbox{shadecolor}{##1}\hskip-\fboxsep
     % There is no \\@totalrightmargin, so:
     \hskip-\linewidth \hskip-\@totalleftmargin \hskip\columnwidth}%
 \MakeFramed {\advance\hsize-\width
   \@totalleftmargin\z@ \linewidth\hsize
   \@setminipage}}%
 {\par\unskip\endMakeFramed%
 \at@end@of@kframe}
\makeatother

\definecolor{shadecolor}{rgb}{.97, .97, .97}
\definecolor{messagecolor}{rgb}{0, 0, 0}
\definecolor{warningcolor}{rgb}{1, 0, 1}
\definecolor{errorcolor}{rgb}{1, 0, 0}
\newenvironment{knitrout}{}{} % an empty environment to be redefined in TeX

\usepackage{alltt}

\usepackage{amssymb,amsmath}
\usepackage{enumerate}
\usepackage{float}
\usepackage{verbatim}
\usepackage{setspace}
\usepackage{multicol}

%% LaTeX margin settings:
  \setlength{\textwidth}{7.0in}
\setlength{\textheight}{9in}
\setlength{\oddsidemargin}{-.5in}
\setlength{\evensidemargin}{0in}
\setlength{\topmargin}{-1.5cm}

%% tell knitr to use smaller font for code chunks
\def\fs{\footnotesize}
\def\R{{\sf R}}
\newcommand{\bfbeta}{\mbox{\boldmath $\beta$}}
\newcommand{\bfD}{\mbox{\boldmath $D$}}
\newcommand{\bfL}{\mbox{\boldmath $L$}}
\newcommand{\bfR}{\mbox{\boldmath $R$}}
\newcommand{\bfmu}{\mbox{\boldmath $\mu$}}
\newcommand{\bfv}{\mbox{\boldmath $V$}}
\newcommand{\bfX}{\mbox{\boldmath $X$}}
\newcommand{\bfy}{\mbox{\boldmath $y$}}
\newcommand{\bfb}{\mbox{\boldmath $b$}}
\newcommand{\ytil}{\mbox{$\tilde{y}$}}
\IfFileExists{upquote.sty}{\usepackage{upquote}}{}
\begin{document}


  
  
  \begin{center}
\large{Sampling: HW7} \\
Leslie Gains-Germain
\end{center}

\begin{doublespacing}

\begin{enumerate}

\item The estimated regression equation I would propose is:
\begin{align*}
\hat{fish} = \hat{\beta_0} + \hat{\beta_1}x_1
\end{align*}
where $\hat{fish}$ is the estimated mean number of fish caught by anglers on this lake in August after fishing $x_1$ hours. $\hat{\beta_0}$ is the estimated mean number of fish caught after fishing $0$ hours (we would expect this estimate to be near $0$). $\hat{\beta_1}$ is the estimated mean number of fish caught per hour by anglers on this lake in August. $\hat{\beta_1}$ is the quantity of interest.

\item The estimated regression equation I would propose is:
\begin{align*}
\hat{p}_{sports} &= \hat{\beta_0} + \hat{\beta_1}x_1 + \hat{\beta_{11}}x_1^2 + \hat{\beta_2}x_2
\end{align*}
where $\hat{p}_{sports}$ is the estimated proportion of time in television news broadcasts in Bozeman that is devoted to sports in month $x_1$, with $x_2$ breaking news events. $x_1$ is the month of the year ($x_1 \in (1, 2,..., 12)$), and $x_2$ is the number of breaking news events that occur each month. I would expect the proportion of time devoted to sports to peak in February because of the superbowl, and then decrease after that. I have included a squared term in the estimated regression equation to allow for this curvature in the relationship. For the second predictor, I would expect months with greater numbers of breaking news events to have lower proportions of time devoted to sports because more time will be devoted to these breaking news events.




\item \begin{enumerate}

\item I fit the three regression models in the code below.

\begin{singlespace}
\begin{knitrout}\footnotesize
\definecolor{shadecolor}{rgb}{0.969, 0.969, 0.969}\color{fgcolor}\begin{kframe}
\begin{alltt}
\hlstd{lm.1} \hlkwb{<-} \hlkwd{lm}\hlstd{(volume} \hlopt{~} \hlstd{diameter,} \hlkwc{data} \hlstd{= cherry)}
\hlstd{lm.2} \hlkwb{<-} \hlkwd{lm}\hlstd{(volume} \hlopt{~} \hlstd{height,} \hlkwc{data} \hlstd{= cherry)}
\hlstd{lm.3} \hlkwb{<-} \hlkwd{lm}\hlstd{(volume} \hlopt{~} \hlstd{diameter} \hlopt{+} \hlstd{height,} \hlkwc{data} \hlstd{= cherry)}
\end{alltt}
\end{kframe}
\end{knitrout}
\end{singlespace}

\item Using the first regression model, the total volume for all black cherry trees in the forest is estimated to be $102318.9$ cubic feet, with a $95\%$ confidence interval from $97709.0$ to $106928.7$ cubic feet. My work is shown below.

\begin{singlespace}
\begin{knitrout}\footnotesize
\definecolor{shadecolor}{rgb}{0.969, 0.969, 0.969}\color{fgcolor}\begin{kframe}
\begin{alltt}
\hlstd{n} \hlkwb{<-} \hlnum{31}
\hlstd{N} \hlkwb{<-} \hlnum{2967}
\hlstd{ybar} \hlkwb{<-} \hlkwd{mean}\hlstd{(cherry}\hlopt{$}\hlstd{volume)}
\hlstd{xbard} \hlkwb{<-} \hlkwd{mean}\hlstd{(cherry}\hlopt{$}\hlstd{diameter)}
\hlstd{txd} \hlkwb{<-} \hlnum{41835}
\hlstd{thaty1} \hlkwb{<-} \hlstd{N} \hlopt{*} \hlstd{ybar} \hlopt{+} \hlkwd{coef}\hlstd{(lm.1)[}\hlnum{2}\hlstd{]} \hlopt{*} \hlstd{(txd} \hlopt{-} \hlstd{N} \hlopt{*} \hlstd{xbard)}
\hlstd{c} \hlkwb{<-} \hlstd{N} \hlopt{*} \hlstd{(N} \hlopt{-} \hlstd{n)} \hlopt{/} \hlstd{n}
\hlstd{mse1} \hlkwb{<-} \hlkwd{anova}\hlstd{(lm.1)[}\hlnum{2}\hlstd{,} \hlnum{3}\hlstd{]}
\hlstd{var.that.y1} \hlkwb{<-} \hlstd{c} \hlopt{*} \hlstd{mse1}
\hlstd{tstar} \hlkwb{<-} \hlkwd{qt}\hlstd{(}\hlnum{0.975}\hlstd{,} \hlnum{29}\hlstd{)}
\hlstd{ci1} \hlkwb{<-} \hlkwd{c}\hlstd{(thaty1} \hlopt{-} \hlstd{tstar} \hlopt{*} \hlkwd{sqrt}\hlstd{(var.that.y1), thaty1} \hlopt{+} \hlstd{tstar} \hlopt{*} \hlkwd{sqrt}\hlstd{(var.that.y1))}
\end{alltt}
\end{kframe}
\end{knitrout}
\end{singlespace}

Using the third regression model, the total volume for all black cherry trees in the forest is estimated to be $52884.3$ cubic feet, with a $95\%$ confidence interval from $38359.7$ to $67408.9$ cubic feet. My work is shown below.

\begin{singlespace}
\begin{knitrout}\footnotesize
\definecolor{shadecolor}{rgb}{0.969, 0.969, 0.969}\color{fgcolor}\begin{kframe}
\begin{alltt}
\hlstd{xbarh} \hlkwb{<-} \hlkwd{mean}\hlstd{(cherry}\hlopt{$}\hlstd{height)}
\hlstd{txh} \hlkwb{<-} \hlnum{201756}
\hlstd{thaty2} \hlkwb{<-} \hlstd{N} \hlopt{*} \hlstd{ybar} \hlopt{+} \hlkwd{coef}\hlstd{(lm.2)[}\hlnum{2}\hlstd{]} \hlopt{*} \hlstd{(txh} \hlopt{-} \hlstd{N} \hlopt{*} \hlstd{xbarh)}
\hlstd{mse2} \hlkwb{<-} \hlkwd{anova}\hlstd{(lm.2)[}\hlnum{2}\hlstd{,} \hlnum{3}\hlstd{]}
\hlstd{var.that.y2} \hlkwb{<-} \hlstd{c} \hlopt{*} \hlstd{mse2}
\hlstd{ci2} \hlkwb{<-} \hlkwd{c}\hlstd{(thaty2} \hlopt{-} \hlstd{tstar} \hlopt{*} \hlkwd{sqrt}\hlstd{(var.that.y2), thaty2} \hlopt{+} \hlstd{tstar} \hlopt{*} \hlkwd{sqrt}\hlstd{(var.that.y2))}
\end{alltt}
\end{kframe}
\end{knitrout}
\end{singlespace}

Using the third regression model, the total volume for all black cherry trees in the forest is estimated to be $93362.5$ cubic feet, with a $95\%$ confidence interval from $89153.9$ to $97571.1$ cubic feet. My work is shown below.

\begin{singlespace}
\begin{knitrout}\footnotesize
\definecolor{shadecolor}{rgb}{0.969, 0.969, 0.969}\color{fgcolor}\begin{kframe}
\begin{alltt}
\hlstd{thaty3} \hlkwb{<-} \hlstd{N} \hlopt{*} \hlstd{ybar} \hlopt{+} \hlkwd{coef}\hlstd{(lm.3)[}\hlnum{2}\hlstd{]} \hlopt{*} \hlstd{(txd} \hlopt{-} \hlstd{N} \hlopt{*} \hlstd{xbard)} \hlopt{+}
                    \hlkwd{coef}\hlstd{(lm.3)[}\hlnum{3}\hlstd{]} \hlopt{*} \hlstd{(txh} \hlopt{-} \hlstd{N} \hlopt{*} \hlstd{xbarh)}
\hlstd{mse3} \hlkwb{<-} \hlkwd{anova}\hlstd{(lm.3)[}\hlnum{3}\hlstd{,} \hlnum{3}\hlstd{]}
\hlstd{var.that.y3} \hlkwb{<-} \hlstd{c} \hlopt{*} \hlstd{mse3}
\hlstd{ci3} \hlkwb{<-} \hlkwd{c}\hlstd{(thaty3} \hlopt{-} \hlstd{tstar} \hlopt{*} \hlkwd{sqrt}\hlstd{(var.that.y3), thaty3} \hlopt{+} \hlstd{tstar} \hlopt{*} \hlkwd{sqrt}\hlstd{(var.that.y3))}
\end{alltt}
\end{kframe}
\end{knitrout}
\end{singlespace}

\item I prefer model $1$ because the proportion of the variability in tree volume explained by the regression on diameter is estimated to be $0.9353$, while the proportion of the variability in tree volume explained by the regression on height is only $0.3579$. As a result, the confidence interval for the total tree volume obtained from model $1$ is narrower than the confidence interval obtained from model $2$. 

\item There is evidence that including both variables provides an improved model for estimating the total volume. There is strong evidence that model $3$ is an improvement over model $1$ (p-value $= 0.014$ from extra-sums-of-squares F-stat $= 6.79$ on $1$ and $28$ df). I also noticed that the adjusted $R^2$ for model $3$ is larger than the adjusted $R^2$ for model $1$. As a result, the confidence interval for the true total tree volume is narrower for model $3$.

\begin{singlespace}
\begin{knitrout}\footnotesize
\definecolor{shadecolor}{rgb}{0.969, 0.969, 0.969}\color{fgcolor}\begin{kframe}
\begin{alltt}
\hlkwd{anova}\hlstd{(lm.1, lm.3)}
\end{alltt}
\begin{verbatim}
## Analysis of Variance Table
## 
## Model 1: volume ~ diameter
## Model 2: volume ~ diameter + height
##   Res.Df RSS Df Sum of Sq    F Pr(>F)
## 1     29 524                         
## 2     28 422  1       102 6.79  0.014
\end{verbatim}
\begin{alltt}
\hlstd{ci3[}\hlnum{2}\hlstd{]}\hlopt{-}\hlstd{ci3[}\hlnum{1}\hlstd{]}
\end{alltt}
\begin{verbatim}
## diameter 
##     8417
\end{verbatim}
\begin{alltt}
\hlstd{ci1[}\hlnum{2}\hlstd{]}\hlopt{-}\hlstd{ci1[}\hlnum{1}\hlstd{]}
\end{alltt}
\begin{verbatim}
## diameter 
##     9220
\end{verbatim}
\end{kframe}
\end{knitrout}
\end{singlespace}

\item Assuming the total diameter and total height of black cherry trees in the forest is $41835$ and $201756$ inches, respectively, we are $95\%$ confident that the true total black cherry tree volume in this forest is between $89153.9$ and $97571.1$ cubic feet.

\end{enumerate}


\end{enumerate}

\end{doublespacing}


\end{document}
